\documentclass[12pt,preprint]{aastex}
%\documentclass{emulateapj}
\usepackage{url}
%\usepackage{natbib}
%\usepackage{xspace}
\def\arcsec{$^{\prime\prime}$}
\bibliographystyle{apj}
\newcommand\degree{{^\circ}}
\newcommand\surfb{$\mathrm{mag}/\square$\arcsec}
\newcommand\Gyr{\rm{~Gyr}}
\newcommand\msun{\rm{M}_\odot}
\newcommand\kms{km s$^{-1}$}
\newcommand\al{$\alpha$}
\newcommand\ha{$\rm{H}\alpha$}
\newcommand\hb{$\rm{H}\beta$}



%\shorttitle{Short Title}
%\shortauthors{Yoachim et al.}

\begin{document}

\title{An Object Oriented Feature-Based Scheduler}


\author{Peter Yoachim, Elahesadat Naghib}
%\altaffiltext{1}{Department of Astronomy, University of Washington, Box 351580,
%Seattle WA, 98195; {yoachim@uw.edu} }

\section{Features}
Features can 

\section{Basis Functions}



\section{Reward and Decision Functions}


\section{Performance Function and Optimization}

\section{Discussion}
In the sims_featureScheduler package, we have implemented features and basis functions as python classes. We have then added a survey class which takes a list of basis functions and weights to compute a reward function. Each filter can then be it's own survey instantiation.

Some advantages of a feature-based scheduler:
\begin{itemize}
    \item{OO-nature of the code makes it possible to include other arbitrary schedulers (e.g., pre-scheduled deep drilling field observations)}
    \item{High spatial resolution features make it possible to easily include dithering options on the scheduling algorithm}
    \item{HEALpixel based features make it easy to compute the cost function efficiently with numpy}
    \item{We can queue multiple observations at a time, thus making it so we do not have to compute the reward function after each individual observation.}
    \item{All our data structures should be picklable, so we can use python multiproccessing to get a factor of 6 speedup computing reward functions by running each survey on it's own processor.}
\end{itemize}




%\begin{figure}
%\epsscale{.5}
%\caption{ \emph{continued}.}
%\end{figure}




\end{document}